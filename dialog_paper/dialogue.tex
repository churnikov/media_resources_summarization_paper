\documentclass{dialogue}

\begin{document}

\begin{otherlanguage}{english}
\begin{center}
{\Large\bfseries{A paper on\\news headline generation}}

\medskip

Surname N. P. (\texttt{user@domain.tld}), Surname N. P. (\texttt{user@domain.tld}),\\Surname N. P. (\texttt{user@domain.tld})

\medskip

Affiliation, City, Country
\end{center}

Your abstract in English: describe your system briefly\medskip

\textbf{Key words:} text summarization, headline generation, Russian language (write your own keywords!)
\end{otherlanguage}

\bigskip

\begin{otherlanguage}{russian}
\begin{center}
{\Large\bfseries{Статья о\\генерации заголовков}}

\medskip

Фамилия И. О. (\texttt{user@domain.tld}), Фамилия И. О. (\texttt{user@domain.tld}),\\Фамилия И. О. (\texttt{user@domain.tld})

\medskip

Организация, город, страна
\end{center}

Аннотация на русском: кратко опишите ваши методы и модели.\medskip

\textbf{Ключевые слова:} автореферирование текстов  (перечислите  ключевые слова)
\end{otherlanguage}

\selectlanguage{english}

\section{Introduction}
Describe (briefly!):
\begin{enumerate}
    \item the task and the dataset
    \item your approach and cite some major works, it was based on
    \item the structure of your paper
\end{enumerate}

\section{System description}
This section is devoted to the detailed description of your contribution. The architectures and the methods should be presented here. Try to make your explanation as clear as possible for those who would desire to reproduce your approach.\footnote{Provide a link to a repo with your code, if it is possible}

\section{Data and training}
In this section describe anything related to the prepossessing of the dataset, pretrained embeddings and language models you used and the details of the training procedure.

\section{Experiments}
This section is devoted to the description of your experiment settings.

\section{Results}
This section presents the results of your experiments.

\section{System and error analysis}
This is an optional section. If you conducted any kind of error analysis and / or hyperparameter search or interpretation, present them in this section.

\section{Related work}
This section is obligatory. In this section describe key papers and ideas in the domain of text summariation and headline summarization and cite the works and implementations, if any, you used. Below you can find an instruction on how to cite a paper in the {\LaTeX} style of Dialogue papers.

\section{Conclusion and future work}
Draw a conclusion and provide some insights on how your approach can be improved.


\subsection{Formul\ae}

What if I want a vector like $\vec{x}$?

\begin{equation}
  \text{similarity} = \cos(\theta) = \frac{\mathbf{A} \cdot \mathbf{B}}{\|\mathbf{A}\| \|\mathbf{B}\|}
\end{equation}

\subsection{Table}

Table 1 is a nice one.

\begin{table}[htbp]
\centering
\caption{Wow!}
\begin{tabular}{|c|c|c|}\hline
a & b & c \\\hline
d & e & f \\\hline
g & h & i \\\hline
\end{tabular}
\end{table}

\subsection{Figure}

Figure 1 makes your paper visually appealing.

\begin{figure}[htbp]
\centering
\includegraphics[scale=.4]{dialogue.eps}
\caption{Great!}
\end{figure}

\subsection{Citations}

Citations are a good thing \cite{Panchenko:18,Sharoff:11}. However, at the present moment we do not support proper {\LaTeX} references. Just use \verb|1| instead of \verb|\ref{fig:example}|. Unfortunately, you have to count references manually.

\begin{itemize}
  \item Do not use any kind of non-breaking spaces, such as \verb|~|, \verb|\,|, etc. Use only the regular ones.
\end{itemize}

\section{Improvements}

Do you have an idea of how to improve the template? Please reach us at\\\texttt{https://github.com/nlpub/dialogue-latex}.

\subsubsection*{Acknowledgements}

Example comes here.

\color{blue}\section*{References}

\makeatletter
\renewcommand{\section}{\@gobbletwo}
\makeatother
\bibliography{\jobname}

\end{document}
