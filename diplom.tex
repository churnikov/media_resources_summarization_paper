\documentclass[aps,%
12pt,%
final,%
oneside,
onecolumn,%
musixtex, %
superscriptaddress,%
centertags]{article} %%
\topmargin=-40pt
\textheight=650pt
\usepackage[english,russian]{babel}
\usepackage[utf8]{inputenc}
%всякие настройки по желанию%
\usepackage[colorlinks=true,linkcolor=blue,unicode=true]{hyperref}
\usepackage{euscript}
\usepackage{supertabular}
\usepackage[pdftex]{graphicx}
\usepackage{amsthm,amssymb, amsmath}
\usepackage{textcomp}
\usepackage[noend]{algorithmic}
\usepackage[ruled]{algorithm}
\selectlanguage{russian}

\begin{document}

\begin{titlepage}
\begin{center}
% Upper part of the page
\textbf{\Large САНКТ-ПЕТЕРБУРГСКИЙ \\ ГОСУДАРСТВЕННЫЙ УНИВЕРСИТЕТ} \\[1.0cm]
\textbf{\large Математико-Механический факультет} \\[0.2cm]
\textbf{\large Кафедра информационно аналитических систем}\\[3.5cm]

% Title
\textbf{\LARGE Суммаризация групп в социальных сетях}\\[1.0cm]
\textbf{\Large Дипломная работа студента 645 группы} \\[0.2cm]
\textbf{\Large Чурикова Никиты Сергеевича} \\[3.5cm]

%supervisor
\begin{flushright} \large
\emph{Научный руководитель:} \\
к.ф. - м.н., доцент \textsc{Графеева Н. Г.}
\end{flushright}
 \begin{flushright} \large
\emph{Рецензент:} \\
Руководитель департамента вычислительной биологии \textsc{Яковлев П. А.}
\end{flushright}
\begin{flushright} \large
\emph{Заведующий кафедрой:} \\
к.ф. - м.н., доцент \textsc{Михайлова Е. Г.}
\end{flushright}
\vfill

% Bottom of the page
{\large {Санкт-Петербург}} \par
{\large {2019 г.}}
\end{center}
\end{titlepage}

% Table of contents
\tableofcontents

\section{Аннотация}
Одной из задач обработки естественного языка является задача суммаризации текста.
Ее целью является уменьшение размера исходного текста без потери ключевой информации.
В данной работе мы решаем схожую проблему, но для информационных ресурсов в социальных сетях.
В частности, необходимо рассмотреть задачу суммаризации текстов и картинок, поскольку
это два основных источника информации. В тексте мы приводим численное обоснование
выбранных методов, а также приводим оценку нашей суммаризации людьми.

\section{Введение}
В современном мире создается все больше и больше информации, которую мы можем потреблять.
Новости, статьи, юмор постоянно меняются и создаются людьми. При таком потоке информации
появляется потребность в инструментах, способных давать как можно больше информации
с минимальными потерями.

При чтении новостей люди, как правило, не идут дальше новостных заголовков \cite{},
для популярных технических статей создают краткие описания описывающие их достижения
и основные моменты \cite{}, а визуальный контент нередко подчиняется единому шаблону.

В данной работе мы показываем, как используя современные достижения в области анализа
данных можно извлекать полезную информацию из новостных ресурсов в социальной сети вконтакте \cite{}.

\subsection{Постановка задачи}
Описать технические составляющие задачи

- Потребность в каких данных была

- Как формулировать задачу суммаризации новостного ресурса
\begin{itemize}
  \item Для текстовых групп -- это выделение ключевых слов и генерация заголовков новостей
  \item Для групп с изображений -- это сбор похожих изображений вместе и показ некоторого
        одного изображения для каждой подгруппы.
\end{itemize}
\subsection{Обзор литературы}
Рассказать про литературу, которая рассматривает задачи выше.
\subsection{Полученные результаты}
Что является результатом работы (будет веб сервис, куда можно закинуть ссылку на группу),
как оценивали качество (продолжить результаты работы алгоритмов толокерам), а также
оценка качества по автоматизированным метрикам, и как они коррелируют с оценками людей.

\section{Алгоритмы, использованные в работе}
\subsection{Текст}
\subsection{Изображения}

\section{Оценки качества}

\section{Эксперименты}

\section{Выводы}

\section{Заключение}


\bibliographystyle{gost780s}
\bibliography{test}

\end{document}
